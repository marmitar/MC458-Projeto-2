\section{Definições}

Um alpinista deve escalar uma parede representada por uma grade $n \times m$, em cada célula $(i, j)$ da grade tem um custo $C_{i, j} > 0$ associado. O alpinista deve começar na base, linha 1, e chegar no topo, linha $n$, tentando minizar o grau de periculosidade (custo total) do caminho. Quando ele se encontra na célula $(i, j)$, com $1 \leq i < n$ e $1 \leq j \leq m$, os únicos movimentos possíveis são para as células $(i+1, j-1)$, $(i+1, j)$ ou $(i+1, j+1)$, se elas fazem parte da parede.

\subsection{Definições Adicionais}

\begin{definition}[Caminho até o Topo]
    Considere um caminho $P = \left(p_1, \ldots, p_k\right)$, sendo $k$ a quantidade de vértices no caminho. Então, $P$ é um \textit{caminho até o topo} se ele termina em alguma célula do topo da parede, ou seja, $p_k = (n, j)$ para algum $1 \leq j \leq m$.
\end{definition}

\begin{definition}[Risco]
    O grau de periculosidade ou risco $R(P)$ de um caminho $P = \left(p_1, \ldots, p_k\right)$ é dado pela soma dos custos de cada célula do caminho, isto é,
    \[
        R(P) = \sum_{1 \leq i \leq k} C_{p_i}
    \]
\end{definition}
